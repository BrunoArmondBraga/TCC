%!TeX root=../tese.tex
%("dica" para o editor de texto: este arquivo é parte de um documento maior)
% para saber mais: https://tex.stackexchange.com/q/78101/183146

%% ------------------------------------------------------------------------- %%
\chapter{Introdução}
\label{cap:introducao}

Árvores Binárias de Busca (ABBs) são estruturas de dados que armazenam um conjunto de chaves de um universo estático, que possui uma ordem total, e dão suporte a buscas neste conjunto. Denotaremos por $n$ o número de elementos do conjunto armazenado na ABB considerada.

Estas estruturas são fundamentais na ciência da computação e possuem as mais diversas finalidades. Nesse projeto iremos estudar a chamada Conjectura da Otimalidade Dinâmica, interpretações geométricas de algoritmos de busca em ABBs e suas delimitações associadas. Também serão estudadas as árvores splay e as árvores tango.

Adaptando a definição dada por Sedgewick e Wayne \cite{Sedgewick}, \textit{árvore binária de busca} é uma árvore binária onde cada nó possui uma chave comparável e possivelmente um valor associado. Além disso, os nós satisfazem a restrição de que a chave em qualquer nó é maior do que as chaves em todos os nós na subárvore esquerda desse nó e menor do que as chaves em todos os nós na subárvore direita desse nó.

As ABBs também possuem um atributo \textit{raiz} que aponta para o nó raiz ou possui valor \textit{nulo}, caso a árvore esteja vazia.

\textit{Rotações} são operações que trocam dois nós, pai-filho, entre si enquanto mantém a restrição vista acima. Veja a Figura~\ref{fig:imagem}. Essa operação é fundamental para garantir performance em alguns algoritmos de ABBs e é muito utilizada para controlar a altura das árvores.


% Arrumar figura!!
\begin{figure}[h]
    \centering
    \includegraphics{imagens/zig2.pdf}
    \caption{Esquema de rotações.}
    \label{fig:imagem}
\end{figure}

No contexto da Conjectura da Otimalidade Dinâmica, apenas serão consideradas buscas bem sucedidas, que chamaremos de \textit{acessos}, e não são consideradas inserções ou remoções no conjunto armazenado. No modelo de computação adotado, um algoritmo de busca em uma ABB, para fazer o acesso a uma chave, mantém um único ponteiro. Chamamos de \textit{nó corrente} o nó apontado por esse ponteiro. 

No início da execução do acesso, o ponteiro aponta para a raiz da árvore e, após uma sequência de operações, deve encontrar o nó que armazena a chave procurada. As operações chamadas de \textit{primitivas} são:
\begin{enumerate}
    \item Mover o ponteiro para o filho esquerdo do nó corrente.
    \item Mover o ponteiro para o filho direito do nó corrente.
    \item Mover o ponteiro para o pai do nó corrente.
    \item Fazer uma rotação que troca a posição do nó corrente e do seu pai.
\end{enumerate}

O algoritmo tradicional de busca em ABB não executa a operação primitiva 4. Ele inicia a execução de um acesso na raiz e desce para o filho apropriado até alcançar a chave procurada. No modelo de computação adotado, todas essas operações possuem custo unitário.

Em geral, durante a execução de um acesso, uma série de nós distintos são apontados pelo ponteiro do algoritmo de busca. Ao fim da execução de uma busca bem sucedida, o nó corrente é o nó procurado. Assim, dizemos que o nó procurado foi \textit{acessado} e que todos os nós que foram apontados pelo ponteiro durante a busca foram \textit{visitados}. O nó acessado também é considerado um nó visitado.

O pior caso acontece quando a chave procurada está em um nó folha mais profundo. Nesse caso o algoritmo tem que passar por toda a altura da árvore até chegar a essa folha. Assim o custo do pior caso é proporcional à altura da árvore, que pode ser em princípio linear em $n$.

Com intuito de mitigar o custo do pior caso, algumas ABBs estrategicamente utilizam da operação primitiva 4 (rotação) para balancear o tamanho das subárvores e assim diminuir a altura da árvore. Um exemplo famoso, sobre o qual daremos mais detalhes à frente, é a árvore splay. Essa ABB executa rotações baseando-se na heurística ``move to front'' e assim balanceia a árvore durante as buscas. Essa árvore não utiliza armazenamento adicional para isso, ou seja, seus nós não têm campos extra.

O custo para executar um acesso nesse modelo está relacionado com o número de operações primitivas executadas até encontrar a chave procurada durante um acesso. 

No artigo de Demaine et al.~\cite{rotation_distance}, é demonstrada a seguinte propriedade sobre ABBs. É possível transformar qualquer ABB com $t$ nós em qualquer outra ABB com os mesmos $t$ nós com no máximo $2t - 6$ rotações. O número preciso não é tão relevante para a análise desse texto. O importante é notar que o número de rotações necessárias para transformar uma ABB $T$ com $t$ nós em uma ABB $T'$ com os mesmos $t$ nós disposta de qualquer outra maneira é $\Oh(t)$, ou seja, é linear no número de nós da ABB.

Por isso, podemos assumir que o número de rotações realizadas durante um acesso é linear no número de nós visitados, assim definiremos o custo para executar um acesso nesse modelo como o número de nós visitados durante esse acesso.

Como o conjunto armazenado nas ABBs que consideraremos não sofre alterações, podemos assumir que esse conjunto é \{1,2,\ldots,$n$\}.

ABBs \textit{online} são ABBs que apenas possuem informações sobre os acessos passados. Ou seja, se $X = (x_1, \ldots, x_m)$ é a sequência de acessos que serão feitos, onde cada $x_i \in \{1,2,\ldots,n\}$, ao acessar $x_{i}$, um algoritmo de busca online não tem conhecimento das chaves $x_{i+1},\ldots,x_{m}$, nem mesmo do valor de $m$. Em particular, muitas ABBs online possuem informações adicionais em cada nó que auxiliam o algoritmo de busca a decidir quando executar rotações durante a busca pelas chaves procuradas. Qualquer ABB online que usa $\Oh(1)$ palavras adicionais por nó possui tempo de execução dominado pelo número de operações primitivas.

Uma ABB é chamada de \textit{offline} se tem conhecimento da sequência $X$ de acessos antes de começar os acessos às chaves de $X$. Nesse contexto, o comportamento do algoritmo não depende unicamente do histórico de acessos passados, uma vez que o algoritmo possui conhecimento prévio de toda a sequência de chaves a serem acessadas.

Dada uma sequência $X$ de acessos, uma ABB é considerada \textit{ótima} se executa os acessos de $X$ com o menor custo possível. Podemos definir OPT$(X)$ como o número de nós visitados por uma ABB ótima para a sequência $X$. Em outras palavras, OPT$(X)$ é o número mínimo necessário de visitas a nós para uma ABB concluir todos os acessos de $X$. 

ABBs balanceadas estabelecem que OPT$(X)$ = $\Oh(m \log n)$, onde $n$ é o número de chaves armazenadas na árvore e $m$ é o comprimento de $X$. Como todo acesso tem custo maior ou igual a 1, então $\text{OPT}(X) \geq m$. Wilber \cite{lowerbound_wilber} provou que OPT$(X)$ = $\Theta$$(m \log n)$ para algumas classes de sequências $X$. 

% JÁ USEI ESSA FRASE!
Uma ABB online é \textit{dinamicamente ótima} se, para todas as sequências $X$, seu algoritmo de busca tem custo $\Oh$(OPT($X$)). De maneira mais geral, uma ABB online é \textit{$c$-competitiva} se executa todos os acessos da sequências $X$ suficientemente longas com custo no máximo $c$\,OPT($X$).

Todo esse estudo foi feito para tentar responder à pergunta: ``\textit{Existe uma ABB online dinamicamente ótima?}".

Uma das tentativas de responder tal pergunta foi a árvore splay de Sleator e Tarjan~\cite{selfadjustingbst}. Como já mencionamos, árvores splay são ABBs que seguem a heurística ``move to front". Mais precisamente, após cada acesso, a árvore se reestrutura de uma maneira particular, trazendo o nó da chave que foi acessada para a raiz da árvore.

Apesar de existirem muitas ABBs muito bem documentadas com tempo por busca logarítmico em $n$, essas estruturas normalmente não conseguem alcançar uma eficiência superior a isso independentemente da entrada. Padrões de acesso do mundo real muitas vezes possuem estruturas repetitivas, como por exemplo bancos de dados que recebem solicitações frequentes para um pequeno número de elementos de alto tráfego. Em alguns desses casos, é possível ter uma performance melhor que $\Theta(\log n)$ por acesso utilizando uma árvore splay, uma vez que essa eventualmente ficaria com as chaves mais acessadas mais próximas à raiz da árvore. 

Há uma conjectura não resolvida no artigo \cite{selfadjustingbst} que diz que as árvores splay são dinamicamente ótimas. Essa conjectura ficou conhecida como a Conjectura da Otimalidade Dinâmica.

A árvore conhecida que mais se aproxima de uma árvore ótima é a árvores tango~\cite{dynamicoptimality}. Tal estrutura é $\Oh(\lg \lg n)$-competitiva. Esta árvore, proposta por Demaine, Harmon, Iacono e Pătrașcu \cite{dynamicoptimality}, utiliza $\Oh(\lg \lg n)$ bits a mais por nó e o custo para determinar a próxima operação primitiva a ser executada é amortizadamente constante.

%Nas árvores tango, há a definição de \textit{filho preferido}. O \textit{filho preferido} de um nó é o filho direito se ele foi o nó mais recentemente acessado em comparação com o nó esquerdo, e caso contrário (incluindo o caso em que nenhum dos dois filhos foi acessado ainda), definimos o nó esquerdo como \textit{filho preferido}.

%Define-se então um \textit{caminho preferido} como um caminho maximal de um nó da ABB até a folha que passa apenas por \textit{filhos preferidos}. 

%Os caminhos preferidos particionam os nós da árvore. Nas árvores tango, os nós de cada caminho preferido são armazenados em uma ABB balanceada tendo como chave a profundidade do nó no caminho. Essas árvores auxiliares estão interligadas de uma maneira apropriada.

%Durante os \textit{acessos}, os \textit{filhos preferidos} podem ser alterados e isso provocará mudanças nas estruturas da árvores auxiliares para manter a estrutura da árvore tango coerentes com o comportamento descrito acima.

Nesse texto, abordaremos uma série de aspectos diferentes em relação a delimitações de custo em algoritmos de busca em ABBs e implementações visando o menor custo possível. No Capítulo 2, será discutido o problema de encontrar uma ABB estática de custo mínimo para uma determinada sequência de acessos. No Capítulo 3, abordaremos o funcionamento e o custo das operações das árvores splay e apresentaremos a Conjectura da Otimalidade Dinâmica. 
No Capítulo 4, examinaremos como tratar buscas em ABBs de maneira geométrica por meio de conjuntos arboreamente satisfeitos e será proposto um algoritmo offline guloso para encontrar conjuntos arboreamente satisfeitos para sequências de entradas. %e será apresentada uma análise de como transforma-lo em um algoritmo online. 
No Capítulo 5, ampliaremos a análise geométrica para desenvolver uma nova delimitação relacionada a retângulos independentes. O Capítulo 6 tratará das delimitações de Wilber e, por fim, no Capítulo 7 explicaremos como funcionam as árvores tango.