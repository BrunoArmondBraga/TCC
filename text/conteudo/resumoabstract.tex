%!TeX root=../tese.tex
%("dica" para o editor de texto: este arquivo é parte de um documento maior)
% para saber mais: https://tex.stackexchange.com/q/78101

% As palavras-chave são obrigatórias, em português e em inglês, e devem ser
% definidas antes do resumo/abstract. Acrescente quantas forem necessárias.
\palavraschave{estrutura de dados, conjectura otimalidade dinâmica, árvores binárias de busca, árvores splay}

\keywords{data structures, dynamic optimality conjecture, binary search trees, splay trees}

% O resumo é obrigatório, em português e inglês. Estes comandos também
% geram automaticamente a referência para o próprio documento, conforme
% as normas sugeridas da USP.
\resumo{
Árvores binárias de busca são estruturas elementares na ciência da computação e sua eficiente capacidade de responder perguntas sobre um conjunto faz com quem elas sejam muito utilizadas e estudadas. Apesar desse vasto estudo, há questões primordiais em aberto sobre o custo ótimo de algoritmos de busca em ABBs para sequências de acesso arbitrárias. Nesse trabalho cobriremos a conjectura da otimalidade dinâmica que diz que árvores splays são assintoticamente tão eficientes quanto qualquer outra árvore para qualquer sequência de acessos. Além disso, cobriremos uma visão geométrica de buscas que é muito útil para caracterizar o custo ótimo e veremos como conseguir delimitar inferiormente custos de sequências de acesso a partir das delimitações da alternância e do funil. Por fim, veremos uma redução que mostra que o problema de múltiplas buscas em árvores binárias de busca é NP-completo.
}

\abstract{
Binary search trees (BSTs) are fundamental structures in computer science, known for their efficient ability to answer queries about a set, making them widely used and studied. Despite extensive research, central questions still open about the optimal cost of search algorithms in BSTs for arbitrary access sequences. In this work, we will address the dynamic optimality conjecture, which states that splay trees are asymptotically as efficient as any other tree for any access sequence. Furthermore, we will explore a geometric approach to searches that proves useful for characterizing the optimal cost and examine how to derive lower bounds for the costs of access sequences based on alternation and funnel bounds. Finally, we will present a reduction demonstrating that the problem of multiple searches in binary search trees is NP-complete.
}
