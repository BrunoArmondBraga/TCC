%!TeX root=../tese.tex
%("dica" para o editor de texto: este arquivo é parte de um documento maior)
% para saber mais: https://tex.stackexchange.com/q/78101/183146

%% ------------------------------------------------------------------------- %%
\chapter{Conclusão}
\label{cap:conclusao}

Esse texto apresenta a Conjectura da Otimalidade Dinâmica assim como uma série de ferramentas que nos auxiliam a entender melhor o comportamento das árvores binárias de busca em contextos mais abstratos.

As árvores splay, cobertas nesse texto, são muito interessantes. A sua heurística simples e fácil de programar nos mostra que mesmo uma ABB que não gasta memória adicional pode ter uma performance assintoticamente eficiente.

É impressionante perceber que é possível traduzir o problema de encontrar o custo ótimo para um modelo de computação em um problema de encontrar superconjuntos com uma propriedade bastante simples da satisfação arbórea. Essa interpretação geométrica é especialmente útil por omitir as rotações sem desconsidera-las, tornando assim o problema  mais visual e intuitivo. 

Além disso, essa abordagem permitiu o desenvolvimento de argumentos espaciais como o de conjuntos independentes de retângulos e o de z-retângulos. Esse foi o passo necessário para conseguir relacionar as duas delimitações propostas por Wilber, cuja conexão permaneceu indeterminada por mais de três décadas.

Por fim, para os leitores interessados, indico o estudo da árvore tango. Essa estrutura de dados, proposta por Demaine, Harmon, Iacono e Pătrașcu  \cite{dynamicoptimality}, possui para qualquer sequência $X$ de acessos custo $\Oh(\lg \lg n) \cdot OPT(X)$ e é a ABB conhecida com custo mais próximo do ótimo. Ela utiliza $\Oh(\lg \lg n)$ bits a mais por nó e funciona simulando a delimitação da alternância e armazenando caminhos preferidos em ABBs balanceadas - onde \textit{caminhos preferidos} são os caminhos maximais que passam apenas por filhos preferidos da ABB. Essa estrutura, apesar de muito interessante, é bastante desafiadora de entender, analisar e principalmente implementar e infelizmente não coube nesse trabalho.