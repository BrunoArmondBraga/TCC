%
% Samuel Plaça de Paula, 2012
% http://www.linux.ime.usp.br/~samuel/mac499/
% samuplaza@gmail.com
%
% Baseado no exemplo disponibilizado por Jesús P. Mena-Chalco:
%
% poster-exemplo (versão minimalista)
% http://www.vision.ime.usp.br/~jmena/stuff/poster-exemplo/
%

\documentclass[final]{beamer} 
\usepackage[size=a1,scale=1.1, orientation=portrait]{beamerposter}
%\usepackage[size=custom,width=70.7,height=100,scale=1.0]{beamerposter} % font scale factor=1.0

\usepackage[english]{babel}
\usepackage[utf8]{inputenc}

% para poder usar imagens eps e psfrag

%\usepackage{epstopdf} 
%\usepackage{epsfig}
\usepackage{graphicx}
\newcommand{\tdots}{\,.\,.\,} % in place of \ldots

\usepackage{tikz}
\usepackage{tikz-qtree}
\usetikzlibrary{matrix,backgrounds, decorations.pathreplacing, automata, arrows}
\usepackage{subfig}



\newcommand{\E}{\Sigma}
\newcommand{\cS}{\mathcal{S}}
\newcommand{\Oh}{\mathcal{O}}
\renewcommand{\emph}[1]{\textbf{#1}}


\newcommand{\PP}{\mbox{P}}
\newcommand{\NP}{\mbox{NP}}
\newcommand{\PL}{\mathit{PL}}
\newcommand{\PLI}{\mathit{PLI}}
\newcommand{\OPT}{\mbox{OPT}}

\newtheorem{teo}{Teorema}[section]  % numerado por section
\newtheorem{lema}[teo]{Lema}        % numerado como teo
\newtheorem{cor}[teo]{Corolário}    % numerado como teo
\newtheorem{fato}[teo]{Fato}        % numerado como teo
\newtheorem{mdef}[teo]{Definição}   % numerado como teo

%\newcommand{\OPT}{\mathrm}

\newcommand{\emptystring}{\varepsilon}
\renewcommand{\baselinestretch}{1.11}

%cardinalidade
\newcommand{\card}[1]
{\left|#1\right|}

\let\:=\colon
\let\epsilon=\varepsilon

\def\({\left(}
\def\){\right)}
\def\<{\langle}
\def\>{\rangle}

% valor (e.g., de uma solução)
\newcommand{\Val}[1]
{\mathrm{val}\(#1\)}


% cores utilizadas para os algoritmos
\usepackage{framed}
\definecolor{azul}{rgb}{0.76471,0.81176,0.91373}  % c3cfe9 -> 195,207,233 -> 0.76471   0.81176   0.91373
\definecolor{lilas}{rgb}{0.83529,0.80784,0.89804} % d5cee5 -> 213,206,229 -> 0.83529   0.80784   0.89804
\definecolor{ops}{rgb}{0.9,0.9,0.9} % d5cee5 -> 213,206,229 -> 0.83529   0.80784   0.89804

\urlstyle{same}

%==The poster style============================================================
\usetheme{poster-exemplo}            % our poster style
%--set colors for blocks (without frame)---------------------------------------
  \setbeamercolor{block title}{fg=dblue,bg=white}
  \setbeamercolor{block body}{fg=black,bg=white}
%--set colors for alerted blocks (with frame)----------------------------------
%--textcolor = fg, backgroundcolor = bg, dblue is the jacobs blue
  \setbeamercolor{block alerted title}{fg=dblue,bg=gray!50}%frame color
  \setbeamercolor{block alerted body}{fg=black,bg=gray!20}%body color
%
%==Title, date and authors of the poster=======================================
\title{Binary search trees and the Dynamic Optimality Conjecture}
\author{Bruno Armond Braga \hspace{80pt} Orientadora: Cristina Gomes Fernandes}
\institute{\vspace{3pt}Computer Science Department,
Institute of Mathematics and Statistics, University of São Paulo\vspace{-15pt}}
%\date{\today}


%==============================================================================
%==the poster content==========================================================
%==============================================================================
\begin{document}
%\vspace*{-15mm}
%--the poster is one beamer frame, so we have to start with:
\begin{frame}[t]
%--to seperate the poster in columns we can use the columns environment
\begin{columns}[t] % the [t] options aligns the columns content at the top
%--the left column-------------------------------------------------------------
\begin{column}{0.35\paperwidth}% the right size for a 3-column layout

	\begin{alertblock}{Introduction}

%Árvores Binárias de Busca (ABBs) são estruturas de dados que armazenam um conjunto de chaves de um universo estático, que possui uma ordem total, e dão suporte a buscas neste conjunto. Denotaremos por n o número de elementos do conjunto armazenado na ABB considerada.\vskip2ex

%Árvore binária de busca é uma árvore binária onde cada nó possui uma chave comparável e possivelmente um valor associado. Além disso, os nós satisfazem a restrição de que a chave em qualquer nó é maior do que as chaves em todos os nós na subárvore esquerda desse nó e menor do que as chaves em todos os nós na subárvore direita desse nó.\vskip2ex

Binary search trees (BSTs) are fundamental structures in computer science and rotations play a crucial role in optimizing their efficiency. 
The Dynamic Optimality Conjecture considers a BST that stores the keys from $1$ to $n$ and is subjected to a sequence of accesses (insertions and deletions are not considered). An \textbf{access} is a search for a key from $1$ to $n$, and the search algorithm is allowed to perform rotations. The splay tree is a BST that rotates in a specific way during accesses. The Dynamic Optimality Conjecture, proposed in 1985 and still open, states that the splay tree is \textbf{dynamically optimal}.
\vskip2ex

%Ainda pouco se sabe em relação ao custo ótimo de sequências de acesso. A pergunta que esse projeto se propõe a tratar é: existe uma ABB que é assintoticamente tão boa quanto todas as outras para qualquer sequência de acessos? Assim, defini-se que uma ABB online é \textbf{dinamicamente ótima} se, para todas as sequências de buscas $Z$, seu algoritmo de busca tem custo proporcional ao custo ótimo e não possui conhecimento prévio sobre essas sequências de buscas.


	\end{alertblock}
	\vskip2ex

	\begin{block}{Computation Model\vspace{5pt}}

  %// definiremos abb como um modelo de computação\vskip2ex
  The following computation model will be adopted. A search algorithm in a BST maintains a single pointer during accesses, called the current node. At the beginning of the execution of each access, the pointer points to the root of the tree. There are four operations called primitives:\vskip1ex
  
    \hspace{35pt}1. Move the pointer to the left child of the current node.

    \hspace{35pt}2. Move the pointer to the right child of the current node.

    \hspace{35pt}3. Move the pointer to the parent of the current node.

    \hspace{35pt}4. Perform a rotation that swaps the position of the current node and its parent.\vskip2ex

    The cost of an access in this model is the total number of visited nodes during that access. Let $Z$ be a sequence of accesses. We denote by $\OPT(Z)$ the cost of a search algorithm that achieves the minimum possible cost to process the accesses of $Z$.
	\end{block}
    \vskip1ex
    \begin{block}{Splay Trees\vspace{5pt}}

  Sleator and Tarjan proposed the splay tree in 1985. This data structure follows the “move to front” heuristic and, after each access, the tree is restructured through double rotations to the root, bringing the node of the accessed key to the tree's root.
\vskip2ex
\begin{figure}
  \includegraphics[scale=1.05]{fullsplay2.pdf}
\end{figure}

This restructuring tends to balance a portion of the tree and eventually decrease its height. In fact, Sleator and Tarjan proved that the amortized cost per access in this structure is~$\Oh(\lg n)$ for access sequences of length $\Omega(n)$.
\vskip2ex

The \textbf{Dynamic Optimality Conjecture} states that splay trees have cost $\Oh(\OPT(Z))$ for every sequence $Z$ of accesses. From the previous analysis, it can be concluded that splay trees have cost $\Oh(\lg n \cdot \OPT(Z))$.
\vskip2ex

\begin{block}{Arborally Satisfied Sets\vspace{5pt}}

  Arborally satisfaction is an essential concept for understanding the geometry behind BST searches.
  A pair of points $\{a,b\}$ from a set $P$ is arborally satisfied if $a$ and $b$ are orthogonally collinear or if there is at least one point from the set $P \setminus \{a,b\}$ that lies within the region bounded by the orthogonal rectangle that has $a$ and $b$ as vertices.
  A set is \textbf{arborally satisfied} if all pairs of points in the set are arborally satisfied. Below are examples of point sets with their arborally unsatisfied point pairs highlighted.
    \vskip2ex
    

    \begin{figure}
    \includegraphics[scale=1.55]{ASS-ABB2.pdf}
    %\caption{À esquerda, um conjunto $P$ de pontos arboreamente satisfeito. À direita, um conjunto $P$ de pontos com dois pares de pontos arboreamente insatisfeitos com seus retângulos destacados.}
    \end{figure}

\end{block}

    \end{block}

% ---------------------------------------------------------------------------- %
\vskip2ex
\vspace{-5pt}
	\begin{block}{Extra\vspace{1pt}}
        %Para mais informações, acesse a página do trabalho:
        For more information, access the repository:

		\textcolor{jblue}{{\url{https://github.com/BrunoArmondBraga/TCC}}}
    
    \vskip1ex
    %Esse projeto contou com o financiamento da FAPESP (n° 2024/04708-2) que foi essencial para o estudo e preparo deste material.

    During the development of this work, the author received financial support from FAPESP -- grant \#2024/04708-2.

	\end{block}

\end{column}


% ---------------------------------------------------------------------------- %
\begin{column}{0.60\paperwidth} 
  \vspace{-5pt}
  \begin{block}{Geometric Approach to Search Algorithms\vspace{5pt}}

    In 2009, Demaine, Harmon, Iacono, Kane, and Pătraşcu developed an innovative way to approach the problem. They proved that every execution of a BST search algorithm represents an arborally satisfied set and that every arborally satisfied set represents the execution of a BST search algorithm.\vskip2ex

    For a sequence $Z = (z_1, z_2, \ldots, z_m)$ of accesses, the \textbf{geometric view of $Z$} is the set of points $\{(z_i,i) : i = 1,\ldots,m\}$. The \textbf{geometric view of the execution of a search algorithm} to satisfy $Z$ is the set of points $(x,i)$ such that $x$ is the key of one of the nodes visited during the search for key $z_i$, for $i = 1,\ldots,m$. \vskip2ex

    \begin{figure}
    \includegraphics[scale=1.7]{conjuntoASS2.pdf}
    %\caption{À esquerda, a visão geométrica de $Z = (3,1,4,2,5)$. À direita, a visão geométrica da execução do algoritmo de busca que não efetua rotações na ABB ao centro. Os pontos vermelhos indicam as chaves do restante dos nós visitados durante os acessos.}
    \caption{On the left, the geometric view of Z=(3,1,4,2,5). On the right, the geometric view of the execution of the search algorithm that does not perform rotations in the BST at the center. The red points indicate the keys of the remaining nodes visited during the accesses.}
    \end{figure}
    
    The cost of an algorithm to satisfy $Z$ is the number of points in the geometric view of its execution. The value of $\OPT(Z)$ is the number of points in a smallest arborally satisfied superset of the geometric view of $Z$.        
    
      \end{block}
      \vskip2ex

      \begin{block}{Greedy Future\vspace{5pt}}
        \begin{minipage}[t]{0.4\textwidth}
          \vspace{3pt}

          Consider the geometric view $P$ of a sequence $Z$ of accesses. The \textbf{greedy future} algorithm produces an arborally satisfied set $P' \supseteq P$ as follows. Initially $P' = P$. Slide a horizontal line from bottom to top and, as you pass through each point of $P$, add points to $P'$ on the line so that the point set of $P'$ on the line or below it 
          is arborally satisfied. 
          \vskip2ex
          In the context of BSTs, the corresponding algorithm visits only the nodes on the path from the root to the node with the accessed key and rearranges these nodes to bring the next nodes to be visited closer to the root. See the figure on the right: in the first column, the greedy future algorithm, in the second column, the access to the searched key and, in the third, the final restructuring of the BST after this access.
        \end{minipage}%
        \hfill
        \begin{minipage}[t]{0.6\textwidth}
            \begin{figure}[H]
                \centering
                \includegraphics[scale=1.3]{gulosoComABB.pdf}
                %\caption{À esquerda, o algoritmo guloso futurista. No meio, o acesso a chave buscada e à direita a reestruturação final da ABB após este acesso.}
            \end{figure}
        \end{minipage}
    \end{block}
    \vskip2ex
	\begin{block}{Wilber Bounds\vspace{5pt}}

%R. Wilber deu os passos pioneiros na delimitação de algoritmos de busca que permitem rotações. O trabalho dele, de 1989, trouxe duas delimitações.
		\begin{columns}[c,totalwidth=0.60\paperwidth]

		\begin{column}{0.47\columnwidth}
      
		\begin{block}{Alternation Bound\vspace{5pt}}

      Consider the sequence $Z$ of accesses applied to a static complete binary search tree in which only leaves have keys.
      The \textbf{preferred child} of a node is the most recently visited child. Wilber proved that the sum of the number of times a preferred child changes during accesses of $Z$
      is a lower bound for the cost of any search algorithm that satisfies $Z$.\vskip2ex

      \begin{figure}[H]
        \centering
        \includegraphics[scale=1.2]{alternation.pdf}
        %\caption{Delimitação para a sequência de acessos $Z = (1,5,3,7,2,6,4)$.}
        \caption{Alternation bound for the sequence $Z = (3,2,1,4,2,3)$ of accesses.}
    \end{figure}

        \vspace{33pt}

        \end{block}
        \end{column}

		\begin{column}{0.5\columnwidth}
      \vspace{-16pt}
		\begin{block}{Funnel Bound\vspace{5pt}}

      Let $P$ be the geometric view of a sequence $Z$ of accesses. For each point, count the number of right-left alternations of the points below it, going from top to bottom,
      whose coordinate $x$ is closest to the coordinate $x$ of the analyzed point.
      The sum of these numbers is a lower bound for the cost of any search algorithm that meets~$Z$.\vskip2ex

      \begin{figure}[H]
        \centering
        \includegraphics[scale=1.04]{funnel.pdf}
        %\caption{Funil do ponto mais acima e suas alternâncias.}
        \caption{Funnel of the highest point and its 4 alternations.}
    \end{figure}


        \end{block}
		\end{column}
		\end{columns}

	\end{block}
	%\vskip2ex


% ---------------------------------------------------------------------------- %
\begin{block}{\vspace{-27pt}Tango Trees\vspace{5pt}}

  %A estrutura dentro do modelo de computação adotado que possui maior eficiência assintótica conhecida é a árvore tango. 
  
  Demaine et al., inspired by the alternation bound, developed in 2007 the tango tree whose cost is $\Oh(\lg \lg n \cdot \OPT(Z))$ for every sequence $Z$ of accesses. This is currently the best-known efficiency in the proposed model. A \textbf{preferred path} is the maximal path that passes only through preferred children from a node to a leaf. The central idea of the tango tree is to store the $\Oh(\lg n)$ nodes of each preferred path in a balanced BST, updating them as the access happen and change the preferred children.
  \vspace{5pt}

  \begin{figure}[H]
    \centering
    \includegraphics[scale=1.3]{tangotree2.pdf}
    %\caption{À esquerda, a árvore de referência para a sequência $Z = (13,11,7,3)$. As cores representam caminhos preferidos. À direita, a própria árvore tango.}
    \caption{On the left, the reference tree for the sequence $Z = (13,11,7,3)$. The colors represent preferred paths. On the right, an actual tango tree.}
\end{figure}

\vskip1ex


\end{block}
\vspace{-2ex}
\vspace{-10pt}
\begin{block}{References}
	\scriptsize{\begin{thebibliography}{99}    
    \vspace{-4pt}

    \bibitem{geometry_of_bst}
    Demaine, Erik and Harmon, Dion and Iacono, John and Kane, Daniel and Pătraşcu, Mihai,  
    ``\textbf{The geometry of binary search trees}'',
    in ACM-SIAM Symposium on Discrete Algorithms, 2009.

    \bibitem{dynamicoptimality}
    Demaine, Erik and Harmon, Dion and Iacono, John and Pătraşcu, Mihai,  
    ``\textbf{Dynamic optimality\,---\,{A}lmost}'',
    in SIAM Journal on Computing, 37(1):240–251, 2007.

    \bibitem{selfadjustingbst}
    Sleator, D. Dominic and Tarjan, R. Endre,
    ``\textbf{Self-Adjusting Binary Search Trees}'',
    in Journal of the ACM, 32(3):652-686, 1985.

    \bibitem{lowerbound_wilber}
    Wilber, Robert, 
    ``\textbf{Lower bounds for accessing binary search trees with rotations}'',
    in SIAM Journal on Computing, 18(1):56–67, 1989.

	\end{thebibliography}}
%\hfill{\includegraphics[height=6cm]{logo-ime.svg}}
\end{block}

\end{column}

\end{columns}
% ---------------------------------------------------------------------------- %
\end{frame}
\end{document}
